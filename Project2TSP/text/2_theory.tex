\subsection{A brief introduction to Graph Theory}
There are a number of different types of graphs, with the simplest being dubbed the \textit{simple} graph.
It is simple in the sense that it is undirected, and at most contains a single edge between two nodes.
Additionally, there exists no edge from a node to itself.
Here, a graph is meant as a simple graph, unless otherwise specified.

\begin{comment}
Graphs are in their simplest form a collection of vertices, and a collection of edges connecting said vertices.
Let $V$ denote the set of vertices, and $E \subseteq \{ uv : u,v \in V \wedge u \neq v \}$.
We can then denote the simple, undirected graph as $G = (V, E)$, which by definition contains to loops from a given vertex $u$ to itself.
It is undirected in the sense that we do not assign a direction \textit{from $u$ to $v$}, but rather simply that there is a connection.
\end{comment}

Associated with a graph is the adjacency matrix $A = (a_{ij}) \in \mathbb{R}^{n \times n}$, defined such that $a_{uv} = 1$ if there exists an edge between nodes $u$ and $v$, and $a_{uv} = 0$ otherwise.
Here, $n = |V|$, i.e.\ the number of nodes in the graph.

With some graphs we additionally define a weight function $w: E \to \mathbb{R}$, assigning a weight to each of the edges.
In a graph of cities, this function might then represent the length of a road connecting two cities.

An important measure when working with the complexity of problems, comes from asymptotic analysis.
For a function $g$, we define $\mathcal{O}(g(n))$ as the set of all functions growing slower than (or equal) to $g$. Formally, we write
\begin{multline}
    \mathcal{O}(g(n)) = \{ f(n) : \exists C > 0 \text{ and } n_0 \in \mathbb{N} \\
    \text{s.t. } |f(n)| \leq C |g(n)| \ \forall n \geq n_0 \}.
\end{multline}
Thus, for instance we have $n \in \mathcal{O}(n^2)$.
Although this notation is formally correct, it is traditionally written $n = \mathcal{O}(n^2)$ \cite[p.~100]{aigner2023discrete}.
Note that this is not a precise measure of growth, but rather an upper bound.

Returning to the graph of cities, a common problem one might encounter is to determine the shortest route connecting two cities.
The number of steps required to solve this problem is bounded by $\mathcal{O}(|E| + |V| \log|V|)$ with Dijkstra's Algorithm \cite{dijkstra}, growing linearithmic with the size of the graph.
In problem complexity, the `hardness' of a problem is typically defined by whether or not there exists an algorithm which solves the problem in polynomial time.
A problem of this difficulty is said to be in class P.
As $n \log n = \mathcal{O}(n^2)$, the shortest path problem is `easily' solved.

An interesting class of problems are those in NP, meaning \textit{non-deterministic polynomial time}.
These problems are defined by there being a polynomial time algorithm for verifying a solution, however not necessarily one to find the solution.
Both Sudoku (of arbitrary board size) and the TSP fall into this category, among many others.

\subsection{Graph Neural Networks}
Graph Neural Networks (GNNs) are a class of deep learning methods designed to operate on graphs.
They do this in various ways in order to utilize the inherent graph structure of the data.
One such method is the Graph Convolutional Network (GCN) \cite{kipf2017semisupervised}, which can be viewed as a generalization of Convolutional Neural Networks (CNNs).

CNNs are designed to operate on euclidean data, which typically consists of 1-D or 2-D grids \cite[p.~326--339]{Goodfellow-et-al-2016}.
A convolutional layer is typically made up of three different stages.
Firstly, there's the convolutional stage where an affine transformation is applied, related to how one would convolve two functions.
Given an input matrix $I$ and a matrix called the kernel, the resulting convolution $S$ is defined by
\begin{equation}
    S_{i,j} = (K * I)_{i,j} = \sum_m \sum_n I_{i - m, j - n} K_{m, n}.
\end{equation}
Typically, the dimension of the kernel is much smaller than that of the input.
In a machine learning context, the kernel typically contains trainable weights, allowing for expressivity in how an element is affected by nearby values.

Next, non-linearity is introduced through an activation function.
A typical choice for CNNs is the rectified linear unit ($\relu$), defined as $\relu(x) = \max\{x, 0\}$.
Finally, there is a pooling layer, where each element is replaced by the result of some operation applied to the neighbouring elements, illustrated in \autoref{fig:cnn_pool}.
This could for instance be max pooling, where the result is the maximal value of the neighbouring elements, or simply the sum.

\begin{figure}[!h]
    \centering
    \resizebox{!}{4cm}{% \documentclass[tikz,border=10pt]{standalone}
% \usetikzlibrary{fit,backgrounds}
% \usepackage{ifthen}

% \begin{document}
\begin{tikzpicture}[>=latex]

% Determine the distance between nodes
\def\gridsize{1}
\def\nnodes{3}
\def\target{\the\numexpr\nnodes-1\relax}
\def\purple{purple!80}
\def\teal{teal!35}
\def\back{purple!35}
\def\back{black!40}

% Create nodes in a 4x4 grid
\foreach \x [count=\xi from 1] in {0,...,\nnodes}
    \foreach \y [count=\yi from 1] in {0,...,\nnodes}
    {
        \node[draw, circle, fill=\teal]  (node-\x-\y) at (\x*\gridsize, \y*\gridsize) {};
    }

% Draw horizontal connections
\foreach \y in {0,...,\nnodes}
    \foreach \x [count=\xi from 1] in {0,...,\target}
    {
        \ifthenelse{\y=\target \AND \( \x=1 \OR \xi=1 \)}{}
        {
            \draw[color=\back] (node-\x-\y) -- (node-\xi-\y);
        }
    }

% Draw vertical connections
\foreach \x in {0,...,\nnodes}
    \foreach \y [count=\yi from 1] in {0,...,\target}
    {
        \ifthenelse{\x=1 \and \( \y=\target \OR \yi=\target \)}{}{
            \draw[color=\back] (node-\x-\y) -- (node-\x-\yi);
        }
    }

% Draw diagonal connections
\foreach \x [count=\xi from 1] in {0,...,\target}
    \foreach \y [count=\yi from 1] in {0,...,\target}
    {
        % \ifthenelse{\(\xi=0 \OR \xi=2\) \AND \x=1}{}{
        %     \draw (node-\xi-\y) -- (node-\x-\yi);
        % }
        \ifthenelse{\( \xi=1 \AND \y=\target \) \OR \( \x=1 \AND \yi=\target \)}{}{\draw[color=\back] (node-\xi-\y) -- (node-\x-\yi);}

        \ifthenelse{ \( \x=1 \AND \y=\target \) \OR \( \xi=1 \AND \yi=\target \) }{}{\draw[color=\back] (node-\x-\y) -- (node-\xi-\yi);}
        % 
        % 
    }

\foreach \x in {0,...,2}
    \foreach \y in {\the\numexpr\nnodes-2\relax,...,\nnodes}
    {
        \ifthenelse{\x=1 \AND \y=\target}{}{
        \draw[->, thick, color=\purple] (node-\x-\y) -> (node-1-\target);
        }
    }

\node (box) [draw=\purple,rounded corners,fit = (node-0-\nnodes) (node-0-\the\numexpr\nnodes-2) (node-2-\the\numexpr\nnodes-2)] {};

\end{tikzpicture}
% \end{document}
}
    \caption{An illustration of a typical convolutional layer. Communication happens between elements within a fixed area.}
    \label{fig:cnn_pool}
\end{figure}

These convolutions are dependent on the structure of the data being both rigid and known.
Arranging the data as a graph, as in \autoref{fig:cnn_pool}, kernel-level communication can be described using a weighted directed graph.
In this way, one gets a sense of how convolutions can be extended to arbitrary graphs, where communication happens across the \nth{1}-order neighbourhood of a node.

GCNs update node features through the propagation rule
\begin{equation}\label{eq:gcn}
    H^{(l+1)} = \sigma \left( \tilde{D}^{-\frac{1}{2}} \tilde{A} \tilde{D}^{-\frac{1}{2}} H^{(l)} W^{(l)} \right),
\end{equation}
where $\tilde{A} = A + I_n$ denotes the adjacency matrix of the graph with added self-edges, $\tilde{D}_{i,i} = \sum_j A_{i,j}$ is the matrix with diagonal elements equal to the outdegree of each node, $H^{(l)}$ and $W^{(l)}$ are respectively the activated values and the trainable weights for the $l^{\text{th}}$ layer, with $H^{(0)}$ equal the input graph.
$\sigma : \mathbb{R} \to \mathbb{R}$ denotes the chosen activation function for the layer, applied elementwise.

Note that $\tilde{D}^{-\frac{1}{2}} \tilde{A} \tilde{D}^{-\frac{1}{2}}$ is a way to normalize the values in $\tilde{A}$, derived from an approximation of spectral graph convolutions.
See \textcite{kipf2017semisupervised} and \textcite{spectralGraph} for a more detailed explanation of this approximation.

An important property of adjacency matrices is that $(A^k)_{i,j}$ is the number of paths from node $i$ to node $j$ of length $k$ \cite[p.~125]{aigner2023discrete}, meaning that repeated applications of \eqref{eq:gcn} results in communication over the $k^{\text{th}}$-order neighbourhood around a node.
One such application around a node is visualized in \autoref{fig:gcn}.

\begin{figure}[!h]
    \centering
    \resizebox{!}{4cm}{% \documentclass[tikz,border=10pt]{standalone}
% \usetikzlibrary{fit,backgrounds}

% \begin{document}

\def\purple{purple!80}
\def\teal{teal!35}
\def\back{purple!35}
\def\back{black!40}


\begin{tikzpicture}[>=latex]
    % \node[draw, circle] (0) at (0.774, 0.357) {};
    % \node[draw, circle] (1) at (-0.599, 0.618) {};
    % \node[draw, circle] (2) at (-0.143, -0.712) {};
    % \node[draw, circle] (3) at (0.156, 0.855) {};
    % \node[draw, circle] (4) at (-0.206, 0.36) {};
    % \node[draw, circle] (5) at (-0.755, -0.266) {};
    % \node[draw, circle] (6) at (0.537, -0.6) {};
    % \node[draw, circle] (7) at (0.931, -0.345) {};
    % \node[draw, circle] (8) at (0.304, -0.132) {};
    % \node[draw, circle] (9) at (-1.0, 0.263) {};

    \node[draw, circle, fill=\teal] (0) at (2.76, 2.11) {};
    \node[draw, circle, fill=\teal] (7) at (3.00, 1.01) {};
    \node[draw, circle, fill=\teal] (3) at (1.80, 2.88) {};
    \node[draw, circle, fill=\teal] (4) at (1.23, 2.11) {};
    \node[draw, circle, fill=\teal] (1) at (0.62, 2.51) {};
    \node[draw, circle, fill=\teal] (5) at (0.38, 1.14) {};
    \node[draw, circle, fill=\teal] (9) at (0.00, 1.96) {};
    \node[draw, circle, fill=\teal] (6) at (2.39, 0.00) {};
    \node[draw, circle, fill=\teal] (2) at (1.33, 0.44) {};
    \node[draw, circle, fill=\teal] (8) at (2.03, 1.35) {};
    
    \foreach \i in {3,4,5,9}
        \draw[->, thick, color=\purple] (\i) to (1);
        
    \begin{scope}[-, color=\back]
        \draw (0) to (7);
        \draw (0) to (3);
        \draw (0) to (8);
        \draw (7) to (6);
        \draw (7) to (8);
        \draw (3) to (4);
        \draw (4) to (9);
        \draw (4) to (8);
        \draw (5) to (9);
        \draw (5) to (2);
        \draw (6) to (2);
        \draw (2) to (8);
    \end{scope}
    % \draw[draw=red, rounded corners=30pt]
    %     ([yshift=0.1cm]1.north west) 
    % --  ([xshift=-0.25cm, yshift=-0.05cm]9.north west)
    % --  ([xshift=-0.02cm, yshift=-0.25cm]5.south)
    % --  ([yshift=-0.1cm]4.south east)
    % --  ([xshift=0.25cm, yshift=0.15cm]3.north east)
    % --  cycle;

    \draw[draw=\purple, rounded corners=15pt]
        ([yshift=0.2cm]1.north west) 
    --  ([xshift=-0.3cm]9.west)
    --  ([xshift=0.1cm, yshift=-0.4cm]5.south west)
    --  ([xshift=0.15cm]4.south east)
    --  ([xshift=0.35cm, yshift=0.2cm]3.north east)
    --  cycle;
\end{tikzpicture}
% \end{document}}
    \caption{An application of a GCN layer, with communication across the \nth{1}-order neighbourhood.}
    \label{fig:gcn}
\end{figure}

In practice, the affine transformation $H^{(l)} W^{(l)}$ in \eqref{eq:gcn} can be replaced with a more complex function, such as a multilayer perceptron (MLP).
This might grant the model more expressivity.
Note that this operates elementwise for each node feature, with shared parameters.

One problem with GCNs is that they do not function on complete graphs.
This stems from the fact that with a fully connected graph, the adjacency matrix contains a $1$ everywhere but the diagonal.
Then, $\tilde{A} = A + I_n$ is a matrix of all ones, resulting in $\tilde{D}_{i,i} = n$.
Multiplying a matrix by $\tilde{D}^{-\frac{1}{2}} \tilde{A} \tilde{D}^{-\frac{1}{2}}$ thus causes all node features to result in the same vector, a concept known as oversmoothing \cite{oversmoothing}.
This can also occur with sparser graphs if too many message passing layers are added.

\subsection{Ising formulations}
The Ising model was originally created in order to model the properties of ferromagnets, although it has since found a number of different applications.
The model considers in the simplest case a string of magnetic moments, each assigned either \textit{up} or \textit{down} spin \cite{isingModel}.
An Ising model thus consists of $N$ spins $s_i = \pm 1$, together with a Hamiltonian
\begin{equation}\label{eq:hamiltonian}
    H(s_1, \ldots, s_N) = - \sum_{i < j} J_{ij} s_i s_j - \sum_{i = 1}^N h_i s_i,
\end{equation}
for $J_{ij}, h_i \in \mathbb{R}$ \cite{IsingFormulations}.

Through the function $f(x) = \frac{1}{2}(x+1)$, the spins can be translated to binary decision variables, such that $x_i = f(s_i) \in \{0, 1\}$.
The Ising model is thus closely related to Quadratic Unconstrained Binary Optimization (QUBO) problems, defined as the programming problem
\begin{equation}\label{eq:QUBO}
\begin{array}{rl}
    \text{maximize} & x^T Q x + c^T x \\
    \text{subject to} & x \in \{0, 1\}^n,
\end{array}
\end{equation}
for a given real matrix $Q$ and vector $c$ \cite[p.~1--7]{PunnenAbrahamP2022TQUB}.
Often, $Q$ is upper triangular, meaning we can rewrite the objective function in \eqref{eq:QUBO} as
\begin{equation}
    \sum_{i \leq j} Q_{ij} x_i x_j + \sum_{i = 1}^n c_i x_i.
\end{equation}
Additionally, maximizing an objective function is equivalent with minimizing the negative of the objective function, meaning we can rewrite \eqref{eq:QUBO} as
\begin{equation}
\begin{array}{rl}
    \text{minimize} & \displaystyle -\sum_{i < j} Q_{ij} x_i x_j - \sum_{i = 1}^n c_i x_i \\
    \text{subject to} & x \in \{0, 1\}^n,
\end{array}
\end{equation}
which is immediately recognizable as the Hamiltonian in \eqref{eq:hamiltonian}.

With this framework, one can encode a number of different optimization and related decision problems.
\textcite{IsingFormulations} formulates Hamiltonians for a number of different NP problems, including all 21 of Karp's NP-Complete problems \cite{Karp2010}, with the intended aim of utilizing quantum adiabatic optimization.

However, relaxing the constraint that $x_i \in \{0, 1\}$ to $x_i \in [0, 1]$, \textcite{Schuetz_2022} notes that these Hamiltonians can serve as differentiable loss functions for a neural network.
With this, one can reduce the need for labeled data in the training process of the neural network, in effect \textit{physics-informing} them.
This is especially helpful when working with NP problems, as labeled data is often in low supply, or missing entirely, due to the complexity of the problems.

\subsubsection{Minimal Vertex Covering}
A vertex cover is a coloring of the nodes in a graph, such that each edge is connected to at least one colored node.
The aim of the Minimal Vertex Covering (MVC) problem is to find a coloring such that the number of colored nodes in minimal.

\textcite{IsingFormulations} formulates this as an Ising problem using a bitstring $x \in \mathbb{R}^n$, where $x_v = 1$ if node $v$ is colored, and $0$ otherwise.
The constraint that each edge must be colored is then enforced through
\begin{equation}\label{eq:edge_cover}
    H_{\alpha} = \sum_{uv \in E} (1 - x_u) (1 - x_v),
\end{equation}
where we have $H_{\alpha} = 0$ if and only if each edge is connected to at least one colored node.
The `minimal' aspect is enforced through appending the term
\begin{equation}
    H_{\beta} = \sum_{u \in V} x_u.
\end{equation}
The total Hamiltonian is then $H = \alpha \cdot H_{\alpha} + \beta \cdot H_{\beta}$, where choosing $\alpha > \beta$ ensures that the global minima is optimal for the problem.

In order to improve computational performance, we typically want to avoid using explicit for-loops.
This stems from the fact that we would rather want to use packages designed for numerical linear algebra, which have been optimized for vector-matrix operations over the last decades.

Letting $\bar{x}_u = 1 - x_u$, we utilize the outer product to get
\begin{equation}
    \bar{X} = \bar{x} \bar{x}^T =
    \begin{bmatrix}
        \begin{matrix}
            \bar{x}_1^2 & \bar{x}_1 \bar{x}_2 \\
            \bar{x}_2 \bar{x}_1 & \bar{x}_2^2
        \end{matrix}
         & \cdots & 
         \begin{matrix}
            \bar{x}_1 \bar{x}_n \\
            \bar{x}_2 \bar{x}_n
         \end{matrix} \\
        \vdots & \ddots & \vdots \\
        \begin{matrix}
            \bar{x}_n \bar{x}_1 & \bar{x}_n \bar{x}_2 
        \end{matrix}
        & \cdots & \bar{x}_n^2
    \end{bmatrix},
\end{equation}
such that $\bar{X}_{uv} = (1 - x_u)(1 - x_v)$.
With a simple graph, i.e.\ a graph which is undirected and contains no parallel edges or self-loops, and $A$ as the adjacency matrix we then have that
\begin{equation}
    \mathds{1}_n \left( A \odot \bar{X} \right) \mathds{1}_n = 2 \sum_{uv \in E} (1 - x_u) (1 - x_v),
\end{equation}
as both $\hat{X}_{uv}$ and $\hat{X}_{vu}$ represent the constraint for a single edge.
Here, $\mathds{1}_n = [1, \ 1, \ \ldots, \ 1] \in \mathbb{R}^n$, and $\odot$ denotes the elementwise Hadamard product.

Thus,
\begin{equation}
    H = \mathds{1}_n^T \left( A \odot \bar{X} \right) \mathds{1}_n + \mathds{1}_n^T x
\end{equation}
is equivalent with the original Hamiltonian with $\alpha = 2$ and $\beta = 1$, satisfying the requirement that $\alpha > \beta$.


\subsubsection{Traveling Salesman Problem}
In a graph $G = (V, E)$, a Hamiltonian cycle is a path which starts and ends in the same node, passing through all other nodes in the graph.
\textcite{IsingFormulations} encodes these constraints in a bitstring $x_{v, i}$ of size $n^2$, where $n = |V|$.
$x_{v, i} = 1$ denotes that the $v^{\text{th}}$ node should be placed in the $i^{\text{th}}$ place in the cycle.
Thus, a viable cycle would be one which includes each vertex exactly once, with exactly one vertex at each place in the cycle.

A bitstring is thus considered feasible if
\begin{equation}
    \sum_{v = 1}^n \left( 1 - \sum_{i = 1}^n x_{v,i} \right)^2 + \sum_{i = 1}^n \left( 1 - \sum_{v = 1}^n x_{v,i} \right)^2 = 0.
\end{equation}
This would be enough if we assume that the graph is \textit{complete}, i.e.\ if there exists an edge between each vertex pair.
However, this might not be the case for a given graph.
Thus, the Hamiltonian associated with guaranteeing a Hamiltonian cycle\footnote{Both are named after Sir William Rowan Hamilton.} is
\begin{equation}\label{eq:H_A}
\begin{split}
    H_{\alpha} =& \sum_{v = 1}^n \left( 1 - \sum_{i = 1}^n x_{v,i} \right)^2 + \sum_{i = 1}^n \left( 1 - \sum_{v = 1}^n x_{v,i} \right)^2 \\
    &+ \sum_{uv \notin E}\sum_{i = 1}^n x_{u,i}x_{v,i+1},
\end{split}
\end{equation}
with the convention that $x_{v,n+1} = x_{v,0}$.

Extending this to also formulate TSP is now simple, by adding a term $H_{\beta}$ defined as
\begin{equation}
    H_{\beta} = \sum_{uv \in E} W_{uv} \sum_{i = 1}^n x_{u,i}x_{v,i+1},
\end{equation}
with $W_{uv}$ being the weight associated with the edge $uv$.
The total Hamiltonian for TSP is then
\begin{equation}\label{eq:Hamilton_TSP}
    H = \alpha \cdot H_{\alpha} + \beta \cdot H_{\beta},
\end{equation}
with coefficients $\alpha,\beta > 0$.
As it should always be favourable to add a heavy edge in order to ensure that the solution is a Hamiltonian cycle, we require that $\beta \max_{uv \in E} W_{uv} < \alpha$.

With graphs, it is typically easier to work with the adjacency matrix.
As the bitstring $x_{v,i}$ only gives the placements of the nodes in a cycle, it is not immediately clear how one is supposed to retrieve the predicted adjacency matrix.
However, we were already close to this when formulating the Hamiltonian, as there exists an edge between $u$ and $v$ if $x_{u,i}x_{v,i+1} > 0$ for some $i$.
We therefore define the predicted adjacency matrix $\hat{A} = 
(\hat{a}_{uv}) \in \mathbb{R}^{n \times n}$ such that
\begin{equation}\label{eq:A_sum}
    \hat{a}_{uv} = \sum_{i = 1}^n x_{u,i} x_{v,i+1}. 
\end{equation}

Again, \eqref{eq:Hamilton_TSP} should ideally be described in terms of matrix operations.
With the bitstring $x_{v,i}$ we associate the matrix $X$ defined as
\begin{equation}
X = 
\begin{bmatrix}
    \begin{matrix}
        x_{1,1} & x_{1,2} \\
        x_{2,1} & x_{2,2}
    \end{matrix}
    & \cdots &
    \begin{matrix}
        x_{1,n} \\ x_{2,n}
    \end{matrix} \\
    \vdots & \ddots & \vdots \\
    \begin{matrix}
        x_{n,1} & x_{n,2}
    \end{matrix}
    & \hdots & x_{n,n}
\end{bmatrix}.
\end{equation}
Then, \eqref{eq:A_sum} corresponds with the matrix multiplication
\begin{equation}
\begin{bmatrix}
    \begin{matrix}
        x_{1,1} & x_{1,2} \\
        x_{2,1} & x_{2,2}
    \end{matrix}
    & \cdots &
    \begin{matrix}
        x_{1,n} \\ x_{2,n}
    \end{matrix} \\
    \vdots & \ddots & \vdots \\
    \begin{matrix}
        x_{n,1} & x_{n,2}
    \end{matrix}
    & \cdots & x_{n,n}
\end{bmatrix}
\begin{bmatrix}
    \begin{matrix}
        x_{1,2} & x_{2,2} \\
        x_{1,3} & x_{2,3}
    \end{matrix}
    & \cdots &
    \begin{matrix}
        x_{n,2} \\ x_{n,3}
    \end{matrix} \\
    \vdots & \ddots & \vdots \\
    \begin{matrix}
        x_{1,n} & x_{2,n} \\
        x_{1,1} & x_{2,1}
    \end{matrix}
    & \cdots & \begin{matrix}
        x_{n,n} \\ x_{n,1}
    \end{matrix}
\end{bmatrix}.
\end{equation}
The second matrix corresponds with shifting the rows of the transpose of $X$, giving us
\begin{equation}
    \hat{A} = X P X^T,
\end{equation}
where $P\in\mathbb{R}^{n \times n}$ is the permutation matrix
\begin{equation}
    P =
    \left[
    \begin{array}{cccc}
         &\SetToWidest{1}  \\
         && \ddots \\
         &&& \SetToWidest{1} \\
         \SetToWidest{1}
    \end{array}
    \right].
\end{equation}

With $W \in \mathbb{R}^{n \times n}$ being the matrix of weights $W_{uv}$, with zeroes where there is not an edge, we can express the edge related terms of \eqref{eq:Hamilton_TSP} as
\begin{equation}
\begin{split}
    \sum_{uv \notin E}\sum_{i = 1}^n x_{u,i}x_{v,i+1} &= \mathds{1}_n^T \left( T \odot \hat{A} \right) \mathds{1}_n, \\
    \sum_{uv \in E} W_{uv} \sum_{i = 1}^n x_{u,i}x_{v,i+1} &= \mathds{1}_n^T \left( W \odot \hat{A} \right) \mathds{1}_n,
\end{split}
\end{equation}
where $T = (t_{uv}) \in \mathbb{R}^{n \times n}$ is the matrix defined through
\begin{equation}
    t_{uv} =
    \begin{cases}
        1 & \text{if } W_{uv} = 0, \\
        0 & \text{otherwise}.
    \end{cases}
\end{equation}

The first terms of \eqref{eq:H_A} are
\begin{equation}
\begin{split}
    \sum_{v = 1}^n \left( 1 - \sum_{i = 1}^n x_{v,i} \right)^2 &= \left( \mathds{1}_n - \mathds{1}_n^T X \right)^2, \\
    \sum_{i = 1}^n \left( 1 - \sum_{v = 1}^n x_{v,i} \right)^2 &= \left( \mathds{1}_n - X\mathds{1}_n \right)^2,
\end{split}
\end{equation}
giving us all the terms needed to efficiently implement \eqref{eq:Hamilton_TSP}. The Hamiltonian for TSP is thus defined through
\begin{equation}
\begin{split}
    H_{\alpha} =&  \left( \mathds{1}_n - \mathds{1}_n^T X \right)^2 + \left( \mathds{1}_n - X\mathds{1}_n \right)^2 \\
    &+ \mathds{1}_n^T \left( T \odot \hat{A} \right)\mathds{1}_n, \\
    H_{\beta} =& \ \mathds{1}_n^T \left( W \odot \hat{A} \right) \mathds{1}_n.
\end{split}
\end{equation}


