Graph Convolution Networks seem promising for a number of combinatorial problems, however it seems likely that they would work better in conjunction with more traditional methods.
Without this, one cannot ensure that the problem constraints are satisfied, meaning the predictions are mostly useless.

Using the Ising model framework seems promising for simpler problems like MVC, as each node only predicts a single value.
In addition, the effect of the predictions from one node on the loss is only affected by its immediate neighbours.
With the TSP formulation, each node effects all others, serving as a complicating factors.
It is then not as simple to learn local behaviour.

The poor result on the Traveling Salesman Problem was not entirely surprising, given the complexity of the problem.
Tracing its origins back to the 1800s \cite{TSP_history}, a lot of effort has gone into solving it efficiently, with no clear solution up to this point.


