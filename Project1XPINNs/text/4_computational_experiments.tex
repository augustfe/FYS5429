\subsection{Advection Equation}
\subsubsection{Problem Formulation}
We firstly consider the one-dimensional linear advection equation, with \textcolor{red}{[$\ldots$ residual discontinuity?]}.
The problem has a simple analytical solution, serving as a sanity check for whether our implementation of the method has any merit.
It is defined by
\begin{align}
    u_x + \alpha \cdot u_t &= 0 \text{ on } \Omega & u(x, 0) &= f \text{ \textcolor{red}{on (?)} } \partial\Omega,
\end{align}
where $\Omega = [-1, 1] \times [0, 1]$, and the initial condition $f$ is defined by
\begin{equation*}
    f(x) =
    \begin{cases}
        1 &\text{if} \, x \in [-0.2, 0.2] \\
        0 &\text{else}.
    \end{cases}
\end{equation*}

We primarily consider the problem with $\alpha = 0.5$, defining a wave travelling right with no diffusion.
We therefore decompose the domain according to the predicted discontinuity.
In order to verify that our method is not simply learning the decomposition itself, we check that the wave is able to travel across the interfaces by setting $\alpha = 1$.

Our network is composed of $6$ hidden layers, each with $20$ nodes, using $\tanh$ as our hidden activation function and no activation for the output layer.
We utilize $2000$ points for the interior, and $200$ evenly spread around the boundary and interfaces. 

\subsubsection{Results}

\subsection{Poisson Equation}
\subsubsection{Problem Formulation}
In this subsection, we consider a two-dimensional Poisson equation with residual discontinuity as described by \cite{XPINN_generalize}.
The problem is a second order linear PDE, and is defined as:
\begin{align}\label{eq:poisson}
    u_{xx}+u_{yy} &= f \text{ on } \Omega &
    u(x,y) &= 0 \text{ on } \partial\Omega,
\end{align}
where $\Omega = [0,1] \times [0,1]$, and the residual is defined as:
\begin{equation*}
    f(x,y)=
    \begin{cases}
        1 &\text{if} \, (x,y)\in [0.25,0.75]\times[0.25,0.75] \\
        0 &\text{else}.
    \end{cases}
\end{equation*}
Using our XPINN methodology, we aim to reproduce the results presented by \cite{XPINN_generalize}.
\subsubsection{Results}

\subsection{Navier-Stokes Equation}
\subsubsection{Problem Formulation}
\subsubsection{Results}
